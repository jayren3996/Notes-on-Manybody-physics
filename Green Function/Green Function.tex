\documentclass[journal=jacsat,manuscript=article]{achemso}

\usepackage[version=3]{mhchem} % Formula subscripts using \ce{}
\usepackage{bm}
\usepackage{amsmath}
\usepackage{amsthm}
\newcommand*\mycommand[1]{\texttt{\emph{#1}}}

\author{Jie Ren}
\email{renjie515@mails.ucas.ac.cn}
\title{Green Function}

\begin{document}
%%%%%%%%%%%%%%%%%%%%%%%%%%%%%%%%%%%%%%%%%%%%%%%%%%%%%%%%%%%%%%%%%%%%%
%% Start the main part of the manuscript here.
%%%%%%%%%%%%%%%%%%%%%%%%%%%%%%%%%%%%%%%%%%%%%%%%%%%%%%%%%%%%%%%%%%%%%
\section{Introduction}
We will focus on systems that:
\begin{itemize}
\item live on $L^{d}$-site lattice,
\item with time-independent Hamiltonian,
\item have translational symmetry,
\item are in ground states or thermal equilibrium states.
\end{itemize}
We treat particle operator on lattice as the fundamental theory. And
then taking the thermodynamics limit to define the continuous field
theory. The Green function can be defined both for both lattice and
field, and the two definitions can be related.

\section{Lattice and Field}

\subsection{Discrete Lattice}

We treat the lattice particle operators and their commutation relation as the fundamental theory. The discussion below is mostly general. We cover both bosonic and fermionic case. They are different by at most a sign difference. In those cases, the upper-level sign represent boson, and the lower-level sign represent fermion. The commutation relation is defined as:
\begin{equation}
	\left[A,B\right]_{\mp}=\begin{cases}
	\left[A,B\right] & bosonic\\
	\left\{ A,B\right\}  & fermionic
	\end{cases}.
\end{equation}
Under this definition, the canonical commutation relation for lattice particle operators are:
\begin{equation}
	\left[c_{i,\alpha},c_{j,\beta}\right]_{\mp}
	=\left[c_{i,\alpha}^{\dagger},c_{j,\beta}^{\dagger}\right]_{\mp}=0,
	\label{eq:zero_commutation}
\end{equation}
and
\begin{equation}
	\left[c_{i,\alpha},c_{j,\beta}^{\dagger}\right]_{\mp}
	=\delta_{ij}\delta_{\alpha\beta}
\end{equation}
In the following, we will omit the commutation relation like \eqref{eq:zero_commutation}. 

Since we only consider systems with translational invariance, so the particle operators in momentum space is more important. The Fourier transformation for particle operators is defined as:
\begin{eqnarray}
	c_{\bm{k}_{n},\alpha} & = & 
	\frac{1}{L^{d/2}}\sum_{j}e^{-i\bm{k}_{n}\cdot\bm{x}_{j}}c_{j,\alpha}\\
	c_{j,\alpha} & = & 
	\frac{1}{L^{d/2}}\sum_{n}e^{+i\bm{k}_{n}\cdot\bm{x}_{j}}c_{\bm{k}_{n},\alpha}
\end{eqnarray}
The coordinates and momentums are discrete:
\begin{eqnarray}
	\bm{x}_{j} & = & a\left(j_{1},\cdots,j_{d}\right),\\
	\bm{k}_{n} & = & \frac{2\pi}{aL}\left(n_{1},\cdots,n_{d}\right),
\end{eqnarray}
where $a$ is the lattice constant. In this way, the commutation relation
on momentum space is still normalized:
\begin{equation}
\left[c_{\bm{k}_{m},\alpha},c_{\bm{k}_{n},\beta}^{\dagger}\right]_{\mp}=\delta_{mn}\delta_{\alpha\beta}.
\end{equation}
When consider the time revolution, the operator in the Heisenberg
picture is
\begin{eqnarray}
c_{i,\alpha}\left(t\right) & = & e^{iHt}c_{i,\alpha}e^{-iHt},\\
c_{\bm{k},\alpha}\left(t\right) & = & e^{iHt}c_{\bm{k},\alpha}e^{-iHt}.
\end{eqnarray}
And the imaginary-time revolution is
\begin{eqnarray}
c_{i,\alpha}\left(\tau\right) & = & e^{H\tau}c_{i,\alpha}e^{-H\tau},\\
c_{\bm{k},\alpha}\left(\tau\right) & = & e^{H\tau}c_{\bm{k},\alpha}e^{-H\tau}.
\end{eqnarray}

\subsection{Field theory}
We are now to define a continuous theory. The procedure is done by taking the thermodynamics limit $L\rightarrow\infty$, the discrete summation and continuous integral are related by:
\begin{eqnarray}
	\sum_{\bm{k}_{m}} & \leftrightarrow & 
	\mathcal{V}\int\frac{d^{d}k}{\left(2\pi\right)^{d}}.\\
	\delta_{\bm{k}\bm{k}'}\mathcal{V} & \leftrightarrow & 
	\left(2\pi\right)^{d}\delta\left(\bm{k}-\bm{k}\right) 
	\label{eq:delta_function}
\end{eqnarray}
The field operator can be defined by summing the particle operators:
\begin{equation}
	\psi_{\alpha}\left(\bm{x}\right)
	=\frac{1}{\sqrt{\mathcal{V}}}\sum_{\bm{k}}e^{i\bm{k}\cdot\bm{x}}c_{\bm{k},\alpha},
\end{equation}
while the momentum is still taken the discrete values. The coordinate is now continuous. So we have a field theory description of the lattice system. Since in the field theory, the thermodynamics limit is always implicitly assumed, one can also substitute the momentum summation by the integral:
\begin{eqnarray}
	\psi_{\alpha}\left(\bm{x}\right) & = & 
	\int\frac{d^{d}k}{\left(2\pi\right)^{d}}e^{i\bm{k}\cdot\bm{x}}\psi_{\bm{k},\alpha}\\
	\psi_{\bm{k},\alpha} & = & \sqrt{\mathcal{V}}c_{\bm{k},\alpha}.
\end{eqnarray}
Here we defined a new field $\psi_{\bm k,\alpha}$, which can be though as the nature generalization of particle operator $c_{\bm k,\alpha}$ in the continuum limit. The factor $\sqrt \mathcal V$ act like a half delta function. To see it explicitly, we calculate the commutation relation, the result is:
\begin{equation}
	\left[\psi_{\bm{k},\alpha},\psi_{\bm{k}',\beta}^{\dagger}\right]_{\mp} 
	=\delta_{\alpha\beta}\delta_{\bm{k}\bm{k}'}\mathcal{V}
  	=\delta_{\alpha\beta}\left(2\pi\right)^{d}\delta\left(\bm{k}-\bm{k}'\right),
\end{equation}
where the second the equation used \eqref{eq:delta_function}. The other commutation relation is:
\begin{eqnarray}
	\left[\psi_{\alpha}\left(\bm{x}\right),\psi_{\beta}^{\dagger}\left(\bm{x}'\right)\right]_{\mp} 
	& = & \frac{1}{\mathcal{V}}\sum_{\bm{k},\bm{k}'}e^{i\left(\bm{k}\cdot\bm{x}-\bm{k}'\cdot\bm{x}'\right)}\left[c_{\bm{k},\alpha},c_{\bm{k}',\beta}^{\dagger}\right]\nonumber \\
 	& = & \frac{1}{\mathcal{V}}\delta_{\alpha\beta}\sum_{\bm{k},\bm{k}'}e^{i\bm{k}\cdot\left(\bm{x}-\bm{x}'\right)}\nonumber \\
 	& = & \delta\left(\bm{x}-\bm{x}'\right).
\end{eqnarray}
This normalization means that for quadratic field summation and integration, they are related by:
\begin{equation}
	\sum_{n,m}F_{\alpha\beta}\left(\bm{p}_{n},\bm{q}_{m}\right)c_{\bm{p}_{n},\alpha}^{\dagger}c_{\bm{q}_{m},\beta}
	=\int\frac{d^{d}p}{\left(2\pi\right)^{d}}\frac{d^{d}q}{\left(2\pi\right)^{d}}F_{\alpha\beta}\left(\bm{p},\bm{q}\right)\psi_{\bm{p},\alpha}^{\dagger}\psi_{\bm{q},\beta},
\end{equation}
and for diagonal case, similarly:
\begin{equation}
	\int\frac{d^{d}k}{\left(2\pi\right)^{d}}F_{\alpha}\left(\bm{k}\right)\psi_{\bm{k},\alpha}^{\dagger}\psi_{\bm{k},\alpha}
 	=\sum_{n}F_{\alpha}\left(\bm{k}_{n}\right)c_{\bm{k}_{n},\alpha}^{\dagger}c_{\bm{k}_{n},\alpha}.
\end{equation}
Time revolution for the field operator is the same as particle operator, but note that sometimes we would define the Matsubara Fourier transformation with the normalization:
\begin{eqnarray}
	\psi\left(i\omega_{n}\right) & = & \frac{1}{\sqrt{\beta}}\int_{0}^{\beta}e^{+i\omega_{n}\tau}\psi\left(\tau\right),\\
	\psi\left(\tau\right) & = & \frac{1}{\sqrt{\beta}}\int_{0}^{\beta}e^{-i\omega_{n}\tau}\psi\left(i\omega_{n}\right).
\end{eqnarray}
this normalization is suitable for path integral calculation of corralation function.


\subsection{Green function and Lehmann representation}

The Green functions is defined by the field operator:
\begin{eqnarray}
	G_{\alpha\beta}^{T}\left(\bm{x},t\right) & = & -i\left\langle T_{t}\left[\psi_{\alpha}\left(\bm{x},t\right)\psi_{\beta}^{\dagger}\right]\right\rangle \\
	G_{\alpha\beta}^{R}\left(\bm{x},t\right) & = & -i\theta\left(t\right)\left\langle \left[\psi_{\alpha}\left(\bm{x},t\right),\psi_{\beta}^{\dagger}\right]_{\mp}\right\rangle \\
	G_{\alpha\beta}^{A}\left(\bm{x},t\right) & = & i\theta\left(-t\right)\left\langle \left[\psi_{\alpha}\left(\bm{x},t\right),\psi_{\beta}^{\dagger}\right]_{\mp}\right\rangle \\
	\mathcal{G}_{\alpha\beta}\left(\bm{x},\tau\right) & = & -i\left\langle T_{\tau}\left[\psi_{\alpha}\left(\bm{x},\tau\right)\psi_{\beta}^{\dagger}\right]\right\rangle 
\end{eqnarray}
The bracket can be interpreted as ground state expectation or thermal
average. In momentum space:
\begin{equation}
	G_{\alpha\beta}\left(\bm{k},t\right) 
	=\int d^{d}x\ e^{-i\bm{k}\cdot\bm{x}}G_{\alpha\beta}\left(\bm{x},t\right).
\end{equation}
It can be shown that Green function in momentum space can be directly
calculated using lattice particle operator:
\begin{eqnarray}
	G_{\alpha\beta}^{T}\left(\bm{k},t\right) & = & 
	-i\left\langle T_{t}\left[c_{\bm{k},\alpha}\left(t\right)c_{\bm{k},\beta}^{\dagger}\right]\right\rangle \\
	G_{\alpha\beta}^{R}\left(\bm{k},t\right) & = & 
	-i\theta\left(t\right)\left\langle \left[c_{\bm{k},\alpha}\left(t\right),c_{\bm{k},\beta}^{\dagger}\right]_{\mp}\right\rangle \\
	G_{\alpha\beta}^{A}\left(\bm{k},t\right) & = & 
	i\theta\left(-t\right)\left\langle \left[c_{\bm{k},\alpha}\left(t\right),c_{\bm{k},\beta}^{\dagger}\right]_{\mp}\right\rangle \\
	\mathcal{G}_{\alpha\beta}\left(\bm{k},\tau\right) & = & 
	-i\left\langle T_{\tau}\left[c_{\bm{k},\alpha}\left(\tau\right)c_{\bm{k},\beta}^{\dagger}\right]\right\rangle 
\end{eqnarray}
In frequency space:
\begin{equation}
	G_{\alpha\beta}\left(\bm{k},\omega\right)
	=\int d^{d}t\ e^{i\omega t}G_{\alpha\beta}\left(\bm{k},t\right).
\end{equation}
It helpful to define two more functions, as they are the building
block of other Green functions:
\begin{eqnarray}
	G_{\alpha\beta}^{>}\left(\bm{x},t\right) & = & -i\left\langle \psi_{\alpha}\left(\bm{x},t\right)\psi_{\beta}^{\dagger}\right\rangle \\
	G_{\alpha\beta}^{<}\left(\bm{x},t\right) & = & \mp i\left\langle \psi_{\beta}^{\dagger}\psi_{\alpha}\left(\bm{x},t\right)\right\rangle \\
	G_{\alpha\beta}^{>}\left(\bm{k},t\right) & = & -i\left\langle c_{\bm{k},\alpha}\left(t\right)c_{\bm{k},\beta}^{\dagger}\right\rangle \\
	G_{\alpha\beta}^{<}\left(\bm{k},t\right) & = & \mp i\left\langle c_{\bm{k},\beta}^{\dagger}c_{\bm{k},\alpha}\left(t\right)\right\rangle 
\end{eqnarray}
We calculate the momentum space function by insert identity resolution into the average, the first function is then:
\begin{eqnarray}
	G_{\alpha\beta}^{>}\left(\bm{k},t\right)\nonumber 
	& = & -\frac{i}{Z}\sum_{nm}\rho_{n}\left\langle n\right|c_{\bm{k},\alpha}\left(t\right)\left|m\right\rangle \left\langle m\right|c_{\bm{k},\beta}^{\dagger}\left|n\right\rangle \nonumber \\
	& = & -\frac{i}{Z}\sum_{nm}\rho_{n}e^{-i\left(\epsilon_{n}-\epsilon_{m}\right)t}C_{\bm{k},\alpha}^{nm}
\end{eqnarray}
The second function is:
\begin{eqnarray}
	G_{\alpha\beta}^{<}\left(\bm{k},t\right)\nonumber
 	& = & \mp\frac{i}{Z}\sum_{nm}\rho_{m}\left\langle m\right|c_{\bm{k},\beta}^{\dagger}\left|n\right\rangle \left\langle n\right|c_{\bm{k},\alpha}\left(t\right)\left|m\right\rangle \nonumber \\
 	& = & \mp\frac{i}{Z}\sum_{nm}\rho_{m}e^{-i\left(\epsilon_{n}-\epsilon_{m}\right)t}C_{\bm{k},\alpha}^{nm},
\end{eqnarray}
where
\begin{equation}
	C_{\bm{k},\alpha}^{nm}=\left\langle n\right|c_{\bm{k},\alpha}\left|m\right\rangle \left\langle m\right|c_{\bm{k},\beta}^{\dagger}\left|n\right\rangle .
\end{equation}
The time ordered Green function can be calculated bu adding up the building blocks:
\begin{eqnarray}
	G_{\alpha\beta}^{T}\left(\bm{k},t\right) & = & \theta\left(t\right)G_{\alpha\beta}^{>}\left(\bm{k},t\right)+\theta\left(-t\right)G_{\alpha\beta}^{<}\left(\bm{k},t\right)\\
	G_{\alpha\beta}^{T}\left(\bm{k},\omega\right) & = & \int\frac{d\omega'}{2\pi}\left[\frac{A\left(\bm{k},\omega'\right)}{\omega-\omega'+i\eta}\mp\frac{B\left(\bm{k},\omega'\right)}{\omega-\omega'-i\eta}\right],
\end{eqnarray}
where the spectrum function $A\left(\bm{k},\omega'\right), B\left(\bm{k},\omega'\right)$ are:
\begin{eqnarray}
	A\left(\bm{k},\omega'\right) & = & \frac{1}{Z}\sum_{nm}\rho_{n}C_{\bm{k},\alpha}^{nm}\delta\left(\omega'-\epsilon_{n}+\epsilon_{m}\right)\\
	B\left(\bm{k},\omega'\right) & = & \frac{1}{Z}\sum_{nm}\rho_{m}C_{\bm{k},\alpha}^{nm}\delta\left(\omega'-\epsilon_{n}+\epsilon_{m}\right)
\end{eqnarray}
The retarded function is:
\begin{eqnarray}
	G_{\alpha\beta}^{R}\left(\bm{k},t\right) & = & \theta\left(t\right)\left[G_{\alpha\beta}^{>}\left(\bm{k},t\right)-G_{\alpha\beta}^{<}\left(\bm{k},t\right)\right]\\
	G_{\alpha\beta}^{R}\left(\bm{k},\omega\right) & = & \int\frac{d\omega'}{2\pi}\frac{A\left(\bm{k},\omega'\right)\mp B\left(\bm{k},\omega'\right)}{\omega-\omega'+i\eta},
\end{eqnarray}
Similarly, the advanced function is
\begin{eqnarray}
	G_{\alpha\beta}^{A}\left(\bm{k},t\right) & = & -\theta\left(-t\right)\left[G_{\alpha\beta}^{>}\left(\bm{k},t\right)-G_{\alpha\beta}^{<}\left(\bm{k},t\right)\right]\\
	G_{\alpha\beta}^{A}\left(\bm{k},\omega\right) & = & \int\frac{d\omega'}{2\pi}\frac{A\left(\bm{k},\omega'\right)\mp B\left(\bm{k},\omega'\right)}{\omega-\omega'-i\eta},
\end{eqnarray}
Using identity:
\begin{equation}
	\frac{1}{\omega\pm i\eta}=\mathcal{P}\frac{1}{\omega}\mp i\pi\delta\left(\omega\right),
\end{equation}
and we have the general structure for real-time Green function. i.e. the real parts of three Green functions are identical: 
\begin{equation}
	Re\ G_{\alpha\beta}^{T}\left(\bm{k},\omega\right)=Re\ G_{\alpha\beta}^{R}\left(\bm{k},\omega\right)=Re\ G_{\alpha\beta}^{A}\left(\bm{k},\omega\right),
\end{equation}
and the imaginary parts of retarded and advanced function are opposite:
\begin{equation}
	Im\ G_{\alpha\beta}^{R}\left(\bm{k},\omega\right)=-Im\ G_{\alpha\beta}^{A}\left(\bm{k},\omega\right).
\end{equation}
For Matsubara function:
\begin{equation}
	\mathcal{G}_{\alpha\beta}\left(\bm{k},\tau\right) 
	=-iG_{\alpha\beta}^{>}\left(\bm{k},\tau=it\right)
 	=-\frac{1}{Z}\sum_{nm}\rho_{n}e^{-\left(\epsilon_{n}-\epsilon_{m}\right)\tau}C_{\bm{k},\alpha}^{nm}
\end{equation}
The imaginary-time Fourier transformation is:
\begin{equation}
	\int_{0}^{\beta}d\tau e^{\left(i\omega_{n}-\epsilon_{n}+\epsilon_{m}\right)\tau}=\frac{\pm e^{-\beta\left(\epsilon_{n}-\epsilon_{m}\right)}-1}{i\omega_{n}-\epsilon_{n}+\epsilon_{m}}
\end{equation}
So we get the Lehmann representation of Matsubara function:
\begin{equation}
	\mathcal{G}_{\alpha\beta}\left(\bm{k},i\omega_{n}\right)=\int\frac{d\omega'}{2\pi}\frac{A\left(\bm{k},\omega'\right)}{i\omega_{n}-\omega'},
\end{equation}
where the spectrum function is
\begin{equation}
	A\left(\bm{k},\omega'\right)=\frac{1}{Z}\sum_{nm}\rho_{n}\left(1\mp e^{-\beta\left(\epsilon_{n}-\epsilon_{m}\right)}\right)C_{\bm{k},\alpha}^{nm}\delta\left(\omega'-\epsilon_{n}+\epsilon_{m}\right).
\end{equation}


\section{Non-interacting System}

Suppose after diagonalization, the lattice Hamiltonian becomes
\begin{equation}
	\hat{H}=\sum_{\sigma}\sum_{\bm{k}}\epsilon_{\sigma}\left(\bm{k}\right)c_{\bm{k},\sigma}^{\dagger}c_{\bm{k},\sigma}.
\end{equation}
Then we can define:
\begin{eqnarray}
	c_{k,\sigma} & = & e^{-i\epsilon_{\sigma}\left(\bm{k}\right)}c_{\bm{k},\sigma}\\
	c_{k,\sigma}^{\dagger} & = & e^{+i\epsilon_{\sigma}\left(\bm{k}\right)}c_{\bm{k},\sigma}^{\dagger}\\
	\psi_{\sigma}\left(x\right) & = & \int\frac{d^{d}k}{\left(2\pi\right)^{d}}e^{+ik\cdot x}c_{\bm{k},\sigma}\\
	\psi_{\sigma}^{\dagger}\left(x\right) & = & \int\frac{d^{d}k}{\left(2\pi\right)^{d}}e^{-ik\cdot x}c_{\bm{k},\sigma}^{\dagger}
\end{eqnarray}
where
\begin{equation}
	k\cdot x=\bm{k}\cdot\bm{x}-\epsilon_{\sigma}\left(\bm{k}\right)t.
\end{equation}


\subsection{Basic correlation function}

The building block of other Green function is:
\begin{eqnarray}
	iG_{\sigma}^{>}\left(\bm{k},t\right) & = & e^{-i\epsilon_{\sigma}\left(\bm{k}\right)t}\left\langle c_{\bm{k},\sigma}c_{\bm{k},\sigma}^{\dagger}\right\rangle
	=e^{-i\epsilon_{\sigma}\left(\bm{k}\right)t}\left(1\pm n_{\bm{k}}\right) \\
	iG_{\sigma}^{>}\left(\bm{k},t\right) & = & \pm e^{-i\epsilon_{\sigma}\left(\bm{k}\right)t}\left\langle c_{\bm{k},\sigma}^{\dagger}c_{\bm{k},\sigma}\right\rangle 
	=\pm e^{-i\epsilon_{\sigma}\left(\bm{k}\right)t}n_{\bm{k}},
\end{eqnarray}
where the particle number is:
\begin{equation}
	n_{\bm{k}}=\frac{1}{e^{\beta\epsilon_{\sigma}\left(\bm{k}\right)}\mp1}.
\end{equation}
For field theory, it can be shown that:
\begin{eqnarray}
	\left\langle \psi_{\bm{k},\sigma}^{\dagger}\psi_{\bm{k}',\sigma'}\right\rangle  & = & \delta_{\sigma\sigma'}\left(2\pi\right)^{d}\delta^{d}\left(\bm{k}-\bm{k}'\right)n_{\bm{k}},\\
	\left\langle \psi_{\bm{k},\sigma}\psi_{\bm{k}',\sigma'}^{\dagger}\right\rangle  & = & \delta_{\sigma\sigma'}\left(2\pi\right)^{d}\delta^{d}\left(\bm{k}-\bm{k}'\right)\left(1\pm n_{\bm{k}}\right),
\end{eqnarray}
two basic green function in field theory are defined as:
\begin{eqnarray}
	iG_{\sigma}^{>}\left(x-x'\right)\nonumber
 	& = & \left\langle \psi_{\sigma}\left(x\right)\psi_{\sigma}^{\dagger}\left(x'\right)\right\rangle \nonumber \\
	& = & \int\frac{d^{d}k}{\left(2\pi\right)^{d}}\int\frac{d^{d}k'}{\left(2\pi\right)^{d}}e^{i\left(k\cdot x-k'\cdot x'\right)}\left\langle \psi_{\bm{k},\sigma}\psi_{\bm{k}',\sigma}^{\dagger}\right\rangle \nonumber \\
 	& = & \int\frac{d^{d}k}{\left(2\pi\right)^{d}}e^{ik\cdot\left(x-x'\right)}\left(1\pm n_{\bm{k}}\right) \\
	iG_{\sigma}^{<}\left(x-x'\right)\nonumber
 	& = & \pm\left\langle \psi_{\sigma}^{\dagger}\left(x'\right)\psi_{\sigma}\left(x\right)\right\rangle \nonumber \\
 	& = & \pm\int\frac{d^{d}k}{\left(2\pi\right)^{d}}\int\frac{d^{d}k'}{\left(2\pi\right)^{d}}e^{i\left(k\cdot x-k'\cdot x'\right)}\left\langle \psi_{\bm{k}',\sigma}^{\dagger}\psi_{\bm{k},\sigma}\right\rangle \nonumber \\
 	& = & \pm\int\frac{d^{d}k}{\left(2\pi\right)^{d}}e^{ik\cdot\left(x-x'\right)}n_{\bm{k}}
\end{eqnarray}
In momentum space:
\begin{eqnarray}
	iG_{\sigma}^{>}\left(\bm{k},t\right) & = & e^{-i\epsilon_{\sigma}\left(\bm{k}\right)t}\left(1\pm n_{\bm{k}}\right),\\
	iG_{\sigma}^{<}\left(\bm{k},t\right) & = & \pm e^{-i\epsilon_{\sigma}\left(\bm{k}\right)t}n_{\bm{k}}.
\end{eqnarray}
This agrees with the result in lattice theory.

\subsection{Time-ordered function}

The time-ordered Green function is
\begin{eqnarray}
	G_{\sigma}^{T}\left(\bm{k},t\right)\nonumber 
 	& = & \theta\left(t\right)G_{\sigma}^{>}\left(\bm{k},t\right)+\theta\left(-t\right)G_{\sigma}^{<}\left(\bm{k},t\right)\nonumber \\
 	& = & -ie^{-i\epsilon_{\sigma}\left(\bm{k}\right)t}\left[\theta\left(t\right)\left(1\pm n_{\bm{k}}\right)\pm\theta\left(-t\right)n_{\bm{k}}\right]\nonumber \\
 	& = & -ie^{-i\epsilon_{\sigma}\left(\bm{k}\right)t}\left[\theta\left(t\right)\pm n_{\bm{k}}\right].
\end{eqnarray}
Use the identity
\begin{eqnarray}
\theta\left(+t\right) & = & -\int\frac{d\omega}{2\pi i}\frac{e^{-i\omega t}}{\omega+i\eta},\\
\theta\left(-t\right) & = & +\int\frac{d\omega}{2\pi i}\frac{e^{-i\omega t}}{\omega-i\eta},
\end{eqnarray}
where $\eta=0^{+}$. The time integral is done by:
\begin{eqnarray}
	-ie^{-i\epsilon_{\sigma}\left(\bm{k}\right)t}\theta\left(t\right)
 	& = & \int\frac{d\omega}{2\pi}\frac{e^{-i\left(\omega+\epsilon_{\sigma}\left(\bm{k}\right)\right)t}}{\omega+i\eta}
 	=\int\frac{d\omega}{2\pi}\frac{e^{-i\omega t}}{\omega-\epsilon_{\sigma}\left(\bm{k}\right)+i\eta}\\
	ie^{-i\epsilon_{\sigma}\left(\bm{k}\right)t}\theta\left(-t\right)
 	& = & \pm\int\frac{d\omega}{2\pi}\frac{e^{-i\left(\omega+\epsilon_{\sigma}\left(\bm{k}\right)\right)t}}{\omega-i\eta}
 	=\pm\int\frac{d\omega}{2\pi}\frac{e^{-i\omega t}}{\omega-\epsilon_{\sigma}\left(\bm{k}\right)-i\eta}
\end{eqnarray}
So the Green function in frequency space is
\begin{equation}
	G_{\sigma}^{T}\left(\bm{k},\omega\right)=\frac{1\pm n_{\bm{k}}}{\omega-\epsilon_{\sigma}\left(\bm{k}\right)+i\eta}\mp\frac{n_{\bm{k}}}{\omega-\epsilon_{\sigma}\left(\bm{k}\right)-i\eta}.
\end{equation}
Using identity:
\begin{equation}
	\frac{1}{\omega\pm i\eta}=\mathcal{P}\frac{1}{\omega}\mp i\pi\delta\left(\omega\right).
\end{equation}
\begin{eqnarray}
	Re\ G_{\sigma}^{T}\left(\bm{k},\omega\right) & = & \mathcal{P}\frac{1}{\omega-\epsilon_{\sigma}\left(\bm{k}\right)}\\
	Im\ G_{\sigma}^{T}\left(\bm{k},\omega\right) & = & -\pi\left(1\pm2n_{\bm{k}}\right)\delta\left(\omega-\epsilon_{\sigma}\left(k\right)\right).
\end{eqnarray}


\subsection{Retarded and advanced function}
The retarded function is:
\begin{eqnarray}
	G_{\sigma}^{R}\left(\bm{k},t\right) 
	=-i\theta\left(t\right)\left[G_{\sigma}^{>}\left(\bm{k},t\right)-G_{\sigma}^{<}\left(\bm{k},t\right)\right]
 	= -ie^{-i\epsilon_{\sigma}\left(\bm{k}\right)t}\theta\left(t\right).
\end{eqnarray}
In frequency space:
\begin{equation}
	G_{\sigma}^{R}\left(\bm{k},\omega\right)=\frac{1}{\omega-\epsilon_{\sigma}\left(\bm{k}\right)+i\eta}.
\end{equation}
The advanced function is:
\begin{eqnarray}
	G_{\sigma}^{A}\left(\bm{k},t\right) 
	= i\theta\left(-t\right)\left[G_{\sigma}^{>}\left(\bm{k},t\right)-G_{\sigma}^{<}\left(\bm{k},t\right)\right]
	= ie^{-i\epsilon_{\sigma}\left(\bm{k}\right)t}\theta\left(-t\right).
\end{eqnarray}
In frequency space:
\begin{equation}
	G_{\sigma}^{A}\left(\bm{k},\omega\right)=\frac{1}{\omega-\epsilon_{\sigma}\left(\bm{k}\right)-i\eta}.
\end{equation}
We see that
\begin{equation}
	ReG_{\sigma}^{R}\left(\bm{k},\omega\right)=ReG_{\sigma}^{A}\left(\bm{k},\omega\right)=\frac{1}{\omega-\epsilon_{\sigma}\left(\bm{k}\right)}.
\end{equation}
\begin{equation}
	ImG_{\sigma}^{R}\left(\bm{k},\omega\right)=-ImG_{\sigma}^{A}\left(\bm{k},\omega\right)=-\pi\delta\left(\omega-\epsilon_{\sigma}\left(\bm{k}\right)\right).
\end{equation}

\subsection{Matsubara function}

For non-interacting system, the Matsubara function and the time-ordered
function is basically identical, by replacing
\begin{eqnarray}
	it & \leftrightarrow & \tau\\
	\omega & \leftrightarrow & i\omega_{n}.
\end{eqnarray}
We will assume $\tau>0$, and the Matsubara function is:
\begin{equation}
	\mathcal{G}_{\sigma}\left(\bm{k},\tau\right)=-e^{-\epsilon_{\sigma}\left(\bm{k}\right)\tau}\left[1\pm n_{\bm{k}}\right]
	\end{equation}
In frequency space:
\begin{eqnarray}
	\mathcal{G}_{\sigma}\left(\bm{k},i\omega_{n}\right)\nonumber
 	& = & -\int_{0}^{\beta}e^{\left(i\omega_{n}-\epsilon_{\sigma}\left(\bm{k}\right)\right)\tau}\left[1\pm n_{\bm{k}}\right]\nonumber \\
 	& = & \frac{1\mp e^{-\beta\epsilon_{\sigma}\left(\bm{k}\right)}}{i\omega_{n}-\epsilon_{\sigma}\left(\bm{k}\right)}\left[1\pm\frac{1}{e^{\beta\epsilon_{\sigma}\left(\bm{k}\right)}\mp1}\right]\nonumber \\
 	& = & \frac{1}{i\omega_{n}-\epsilon_{\sigma}\left(\bm{k}\right)}.
\end{eqnarray}

\end{document}
