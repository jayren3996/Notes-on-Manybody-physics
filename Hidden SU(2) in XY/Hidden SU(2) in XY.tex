\documentclass[aps,prb,superscriptaddress]{revtex4}
\usepackage{amsfonts}
\usepackage{amsmath}
\usepackage{amssymb}
\usepackage{graphicx}
\usepackage{bm}
\usepackage{color}
\usepackage{mathrsfs}
\usepackage[colorlinks,bookmarks=true,citecolor=blue,linkcolor=red,urlcolor=blue]{hyperref}
\usepackage{appendix}
\usepackage{float}
\setlength{\parindent}{10 pt}
\setlength{\parskip}{2 pt}
\setcounter{MaxMatrixCols}{30}
\bibliographystyle{apsrev}
\newcommand{\RNum}[1]{\uppercase\expandafter{\romannumeral #1\relax}}

\begin{document}
\title{Hidden SU(2) in Spin-1 XY model}
\author{Jie Ren}
\maketitle

\section{Onsite SU(2)}

For spin-1 model, we can define another onsite SU(2) operator:
\begin{equation}
\begin{cases}
\tilde{s}^{\pm}\equiv\frac{1}{2}\left(S^{\pm}\right)^{2}\\
\tilde{s}^{z}\equiv\frac{1}{2}S^{z}
\end{cases},
\end{equation}
where $S_{j}^{\pm}$ is the regular $S=1$ spin operator. For spin-1:
\begin{eqnarray}
\left[\tilde{s}^{+},\tilde{s}^{-}\right] & = & 2\tilde{s}^{z},\\
\left[\tilde{s}^{z},\tilde{s}^{\pm}\right] & = & \pm\tilde{s}^{\pm},\\
\left\{ \tilde{s}^{+},\tilde{s}^{-}\right\}  & = & \left(S^{z}\right)^{2}.
\end{eqnarray}
The Casimir invariant for this SU(2) is
\begin{equation}
C_{2}=\frac{1}{2}\left(\tilde{s}^{+}\tilde{s}^{-}+\tilde{s}^{-}\tilde{s}^{+}\right)+\left(\tilde{s}^{z}\right)^{2}=\frac{3}{4}\left(S^{z}\right)^{2}.
\end{equation}
So we have
\begin{equation}
\left[\left(S^{z}\right)^{2},\tilde{s}^{\pm}\right]=0.
\end{equation}
Note that
\begin{eqnarray}
\left[\tilde{s}_{j}^{+},S_{j}^{-}\right] & = & S_{j}^{-}\left(1-2\left(S_{j}^{z}\right)^{2}\right),\\
\left[\tilde{s}_{j}^{-},S_{j}^{+}\right] & = & S_{j}^{+}\left(1-2\left(S_{j}^{z}\right)^{2}\right),\\
\left\{ \tilde{s}_{j}^{+},S_{j}^{-}\right\}  & = & S_{j}^{-},\\
\left\{ \tilde{s}_{j}^{-},S_{j}^{+}\right\}  & = & S_{j}^{+}.
\end{eqnarray}


\section{Chain operator }
Define a chain operator
\begin{equation}
U_{j}=\prod_{l=1}^{j-1}\left(1-2\left(S_{l}^{z}\right)^{2}\right).
\end{equation}
Note that
\begin{equation}
S_{j}^{\pm}U_{k}=\begin{cases}
-U_{k}S_{j}^{\pm} & j<k\\
U_{k}S_{j}^{\pm} & j\ge k
\end{cases}.
\end{equation}
We introduce new operators
\begin{equation}
s_{j}^{\pm}=\tilde{s}_{j}^{\pm}U_{j},\ s_{j}=\tilde{s}_{j}.
\end{equation}
Since $\left[U_{j},s_{k}^{\pm}\right]=0$, $\left\{ s_{j}^{z},s_{j}^{\pm}\right\} $
still satisfies su(2) algebra, and we can further define a global
operator:
\begin{equation}
s_{T}^{\pm}=\sum_{j=1}^{L}s_{j}^{\pm},\ s_{T}^{z}=\sum_{j=1}^{L}s_{j}^{z},
\end{equation}
which also satisfies su(2) algebra.



\section{Commutation relation}
The Hamiltonian for spin-1 XY model (with open boundary) is:
\begin{equation}
H_{XY}=\frac{1}{2}\sum_{j=1}^{L-1}\left(S_{j}^{+}S_{j}^{-}+S_{j}^{-}S_{j}^{+}\right).
\end{equation}
We first note that for $k\ne j,j+1$:
\begin{equation}
\left[s_{k}^{+},S_{j}^{+}S_{j+1}^{-}+S_{j}^{-}S_{j+1}^{+}\right]=0.
\end{equation}
While for $k=j$:
\begin{eqnarray}
\left[s_{j}^{+},S_{j}^{+}S_{j+1}^{-}+S_{j}^{-}S_{j+1}^{+}\right] & = & S_{j+1}^{+}\left[\tilde{s}_{j}^{+},S_{j}^{-}\right]U_{j}\\
 & = & S_{j}^{+}S_{j+1}^{+}U_{j+1},
\end{eqnarray}
and for $k=j+1$:
\begin{eqnarray}
\left[s_{j+1}^{+},S_{j}^{+}S_{j+1}^{-}+S_{j}^{-}S_{j+1}^{+}\right] & = & -S_{j}^{+}\left\{ \tilde{s}_{j+1}^{+},S_{j+1}^{-}\right\} U_{j+1}\\
 & = & -S_{j}^{+}S_{j+1}^{+}U_{j+1}.
\end{eqnarray}
In this way
\begin{equation}
\left[s_{T}^{+},H_{XY}\right]=0.
\end{equation}
Similarly, we can show
\begin{equation}
\left[s_{T}^{-},H_{XY}\right]=0.
\end{equation}


%\begin{figure}[H]
%\begin{centering}
%\includegraphics[width=.4\linewidth]{}
%\par\end{centering}
%\caption{caption}
%\end{figure}



\end{document}


