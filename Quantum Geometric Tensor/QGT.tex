\documentclass[aps,prb,superscriptaddress]{revtex4}
\usepackage{amsfonts}
\usepackage{amsmath}
\usepackage{amssymb}
\usepackage{graphicx}
\usepackage{bm}
\usepackage{color}
\usepackage{mathrsfs}
\usepackage[colorlinks,bookmarks=true,citecolor=blue,linkcolor=red,urlcolor=blue]{hyperref}
\usepackage{appendix}
\usepackage{float}
\setlength{\parindent}{10 pt}
\setlength{\parskip}{2 pt}
\setcounter{MaxMatrixCols}{30}
\bibliographystyle{apsrev}
\newcommand{\RNum}[1]{\uppercase\expandafter{\romannumeral #1\relax}}

\begin{document}
\title{Quantum Geometric Tensor}
\author{Jie Ren}
\maketitle


\section{Distance of two quantum states}
Consider a family of parameter-dependent Hamiltonian $H(\lambda)$ smoothly depend on $\lambda = (\lambda_1,\lambda_2,\cdots, \lambda_n)$. The eigen system of the Hamiltonian is $\{E_n(\lambda), \psi_n(\lambda)\}$. A somple definition of the distance between two infinitesimally-varied states is:
\begin{equation}
	ds^2 \equiv \Vert \psi(\lambda+d\lambda)-\psi(\lambda) \Vert^2
	= \langle\partial_\mu \psi | \partial_\nu \psi \rangle d\lambda^\mu d\lambda^\nu
	= (\gamma_{\mu\nu}+i\sigma_{\mu\nu})d\lambda^\mu d\lambda^\nu.
\end{equation}
Because the matrix $ \langle\partial_\mu \psi | \partial_\nu \psi \rangle$ is Hermitian,
\begin{equation}
	\gamma_{\mu\nu} = \gamma_{\nu\mu},\ \sigma_{\mu\nu}=-\sigma_{\nu\mu}.
\end{equation}
The $\sigma_{\mu\nu} d \lambda^\mu d \lambda^\nu$ term vanishes due to the antisymmetry of $\sigma_{\mu\nu}$ and symmetry of $d \lambda^\mu d\lambda^\nu$. So that the distance is
\begin{equation}
	ds^2 = \gamma_{\mu\nu} d\lambda^\mu d\lambda^\nu.
\end{equation}
However, the $\gamma_{\mu\nu}$ is not invariant under the U(1) gauge transformation:
\begin{equation}
	|\psi\rangle \longrightarrow e^{i\alpha(\lambda)}|\psi\rangle,
\end{equation}
which introduce new term
\begin{equation}
	\gamma_{\mu\nu} \longrightarrow \gamma_{\mu\nu}-\beta_\mu \partial_\nu \alpha - \beta_\nu \partial_\mu \alpha + \partial_\mu \alpha \partial_\nu \alpha,
\end{equation}
where $\beta_\mu = i \langle \psi(\lambda)|\partial_\mu \psi \rangle$ is the Berry connection. We can use the Berry connection to define a gauge-invariant metric:
\begin{equation}
	g_{\mu\nu}(\lambda) \equiv \gamma_{\mu\nu}(\lambda) - \beta_\mu(\lambda) \beta_\nu(\lambda).
\end{equation}
Physically, the metric $\gamma_{\mu\nu}$ measures the distance between the ``bare" states in Hilbert space, while the metric $g_{\mu\nu}$ measures the distance of rays in projected Hilbert space $\mathcal{PH} = \mathcal H /U(1)$. For simplicity, we define \textit{Quantum Geometric Tensor}(QGT) $Q_{\mu\nu}$ as
\begin{equation}
	Q_{\mu\nu}(\lambda) \equiv 
	\langle \partial_\mu \psi(\lambda) | \partial_\nu \psi(\lambda) \rangle - 
	\langle \partial_\mu \psi(\lambda) | \psi(\lambda) \rangle 
	\langle \psi(\lambda) | \partial_\nu \psi(\lambda) \rangle.
\end{equation}  
The metrics previously defined are related to QGT by
\begin{equation}
	g_{\mu\nu} = \mathrm{Re}\ Q_{\mu\nu},\ \sigma_{\mu\nu} = \mathrm{Im}\ Q_{\mu\nu}.
\end{equation}

\section{Inner product of two nearby states}
Now consider the inner product of two near by states $|\langle \psi(\lambda+d\lambda) | \psi(\lambda) \rangle|$. Since
\begin{equation}
	\langle \psi(\lambda+d\lambda)| \psi(\lambda)\rangle
	= 1 + i\beta_\mu(\lambda)+\frac{1}{2}\mathrm{Re}\langle \partial_\mu \partial_\nu \psi(\lambda)| \psi(\lambda) \rangle d\lambda^\mu d\lambda^\nu.
\end{equation}
Note that in the third term, we ignore the imaginary part for the same reason as for $\sigma_{\mu\nu}$. Also, since $\langle \partial_\mu| \psi\rangle$ is purely imaginary, differentiate by $\nu$ on both side and take the real part, we have
\begin{equation}
	\mathrm{Re}\langle \partial_\mu \partial_\nu \psi(\lambda)| \psi(\lambda) \rangle
	= -\mathrm{Re}\langle \partial_\mu \psi(\lambda)| \partial_\nu \psi(\lambda) \rangle
	= -\gamma_{\mu\nu}.
\end{equation}  
The absolute value of the inner product is then:
\begin{equation}
	|\langle \psi(\lambda+d\lambda)| \psi(\lambda)\rangle|
	= 1-\frac{1}{2}(\gamma_{\mu\nu}(\lambda)-\beta_\mu(\lambda)\beta_\nu(\lambda)) d\lambda^\mu d\lambda^\nu
	= 1-\frac{1}{2} g_{\mu\nu} d\lambda^\mu d\lambda^\nu.
\end{equation}

\section{Relation with Berry Curvature}
For adiabatic system, $H(\lambda)|\psi_0(\lambda)\rangle = E_0(\lambda)|\psi_0(\lambda)\rangle$. Take derivative with respect to $\mu$ and calculate inner product with $|\psi_n\rangle$($n \ne 0$):
\begin{equation}
	\langle \psi_n |\partial_\mu \psi_0 \rangle = \frac{\langle \psi_n |\partial_\mu H| \psi_0 \rangle}{E_0 - E_n},\ n \ne 0.
\end{equation}
The QGT is
\begin{eqnarray}
	Q_{\mu\nu} &=& \langle \partial_\mu \psi_0 |(1-|\psi_0 \rangle \langle \psi_0|)| \partial_\nu \psi_0 \rangle \nonumber \\
	&=& \sum_{n \ne 0} \langle \partial_\mu \psi_0|\psi_n\rangle \langle\psi_n|\partial_\nu \psi_0 \rangle \nonumber \\
	&=& \sum_{n \ne 0} \frac{\langle \psi_0|\partial_\mu H|\psi_n\rangle \langle\psi_n|\partial_\nu H|\psi_0 \rangle}{(E_0 - E_n)^2}.
\end{eqnarray}
Note that the Berry curvature is
\begin{eqnarray}
	F_{\mu\nu} = \partial_{[\mu,}\beta_{\nu]} = i(Q_{\mu\nu} - Q_{\nu\mu})
	= -2\mathrm{Im}Q_{\mu\nu}.
\end{eqnarray}
So that the imaginary part of QGT is proportional to Berry curvature:
\begin{eqnarray}
	Q_{\mu\nu} = g_{\mu\nu}-\frac{i}{2}F_{\mu\nu}.
\end{eqnarray}

%\begin{figure}[H]
%\begin{centering}
%\includegraphics[width=.4\linewidth]{}
%\par\end{centering}
%\caption{caption}
%\end{figure}



\end{document}


